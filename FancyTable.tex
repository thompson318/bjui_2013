%\definecolor{latextbl}{RGB}{78,130,190}
%this was, which tended to stick the table at teh very end of the document \begin{table*}[h]
\begin{table*}[ht]
%\centering
%\setlength{\extrarowheight}{1pt}
\begin{tabular}{|>{\columncolor[RGB]{78,130,190}}p{0.30\linewidth}>{\columncolor[RGB]{211,203,239}}p{0.30\linewidth}>{\columncolor[RGB]{78,130,190}}p{0.30\linewidth}|}
\hline
%\rowcolor[RGB]{78,130,190}\multicolumn{1}{|>{\columncolor[RGB]{78,130,190}}c}{\color{white}\textbf{System Parameters }} & \multicolumn{1}{>{\columncolor{latextbl!25}}{c}}{\color{black}\textbf{Requirements}} \newline  {\color{red}\textbf{ Performance Measures}} & \multicolumn{1}{>{\columncolor[RGB]{78,130,190}}c|}{\color{white}\textbf{Outcomes}} \\
%{\color{white}\textbf{System Parameters }} & {\color{black}\textbf{Requirements}} \newline  {\color{red}\textbf{ Performance Measures}} & {\color{white}\textbf{Outcomes}} \\
{\color{black}\textbf{System Parameters }} &  {\color{black}\textbf{Design Goals}} & {\color{black}\textbf{Clinical Outcomes}} \\ \hline
%\rowcolor{latextbl!25}
%\begin{enumerate}
%\item{Accuracy}
%\item{Update Rate}
%\item{Visualisation Design}
%\item{User Interface Design }
%\item{Preoperative Imaging Resolution }
%\item{Preoperative Imaging Tissue Contrast }
%\end{enumerate}
\multicolumn{3}{|>{\columncolor[RGB]{211,203,239}}{l}|}{ \textbf{Measures} }\\ 
$\rightarrow$Accuracy \newline
$\longrightarrow$Preop. image resolution \newline
$\longrightarrow$Preop. image distortion \newline
$\longrightarrow$Preop. image contrast \newline
$\longrightarrow$Delay between image and surgery \newline
%$\longrightarrow$Surface error \newline
%$\longrightarrow$Model Errors \newline
$\rightarrow$Update rate \newline
$\rightarrow$Visualisation design \cite{pap271} \newline
$\rightarrow$User interface design 
& 
$\rightarrow$Tumour location\newline
$\rightarrow$Bladder/Prostate Interface\newline
$\rightarrow$Extent of Prostate Capsule\newline
$\rightarrow$Show rectum\newline
$\rightarrow$Show Neuro Vascular Bundles\newline
$\rightarrow$Aid Pre-Op. Planning
&
$\rightarrow$Positive margin rate\newline
$\rightarrow$Biochemical PSA Reccurence\newline
$\rightarrow$Urinary Continence (Months, \%) \newline
$\rightarrow$Erectile FUnction (Months, \%) \newline
$\rightarrow$Damage to rectum \newline
$\rightarrow$Conversion to open \newline
$\rightarrow$Post-Op. Pain \newline
$\rightarrow$Length of hospital stay \newline
$\rightarrow$Improved Training
$\rightarrow$Conversion \newline
$\rightarrow$Survival \newline
\\ \hline
\multicolumn{3}{|>{\columncolor[RGB]{211,203,239}}{l}|}{ \textbf{Measurement Methods} }\\

Direct measurement and laboratory experiment & Observation of system in use and user questionnaire            & Analysis of trial results                                                  \\ \hline
\multicolumn{3}{|>{\columncolor[RGB]{211,203,239}}{l}|}{ \textbf{Development Stage \cite{pap264}}}\\
1 Idea, 2a Development                       & 2a Development, 2b Exploration                                 & 2b Exploration, 3 Assessment, 4 Long-term Study                            \\
\hline
\end{tabular}

\caption{\label{tab:fancyTable}An image guided liver surgery system is defined by the system 
parameters in the left most column. It is reasonable to expect that these will change 
significantly during system development. Further, but less significant changes, can be 
expected after release of the system. However, the system parameters are not of interest 
clinically. The success of failure of the system will be judged by the outcomes in the 
right hand column. A key requirement for an effective development process is 
therefore to link the system parameters with the outcomes. Whilst the system parameters can 
in theory be quantified via experiment or measurement, the system outcomes cannot be 
assessed without using the system on a significant number of patients. The outermost columns can
be linked by the careful assignment of system design goals. If these can be
measured, even subjectively, during system development the development 
cycle can be significantly shortened.}
\end{table*}

