\section{Introduction}
\begin{figure}
\begin{center}
\includegraphics[width=0.23\textwidth]{/home/thompson/phd/data/Exp03_OnPatient/Patient05/AlignAndOverlay/Frame43051_20}
\includegraphics[width=0.23\textwidth]{/home/thompson/phd/data/Exp03_OnPatient/Patient05/AlignAndOverlay/Frame43051}
\end{center}
\caption{\label{fig:Overlay}The surgeon's view of the image guidance system. A preoperative 
image of the patient is overlaid onto the surgical scene. The opacity of the overlay can 
be varied between 0 and 100\%. (20\% shown at left, 100\% at right).}
\end{figure}

The introduction of image guidance to laparoscopic radical prostatectomy is 
an area of increasing interest.
This paper describes our recent experience in implementing a simple image guidance
system in theatre and initial use on 7 patients. 
It became clear during early use of the system that we lacked
a rigorous way to evaluate the system performance. Without a rigorous approach 
to system evaluation it is not possible to show clinical benefit nor 
develop the system to improve patient outcome. The intent of this paper 
is therefore two fold. The paper begins with a description of the 
image guidance system as implemented to date. The second part of the paper
attempts to define an evaluation protocol that will enable proper development of the
system.



