\documentclass[3p,twocolumn,preprint,10pt]{elsarticle}
%these are for FancyTable.tex
\usepackage{colortbl} % for rowcolor
\usepackage{xcolor} % for definecolor
\usepackage{array} %for extrarowheight
\usepackage{glossaries}
\input{/home/thompson/phd/all/chapters/myacronyms}
\include{myacronyms}
\journal{European Urology}
\newcommand{\n}{5}
\newcommand{\Tpreop}{$\mathcal{T}_{MRI \Rightarrow Patient}$}
\newcommand{\Tlens}{$\mathcal{T}_{Patient \Rightarrow Screen}$}
\newcommand{\Tsystem}{$\mathcal{T}_{MRI \Rightarrow Screen}$}
\begin{document}
\begin{frontmatter}
\title{Design and Evaluation of an Image Guidance System for Robot Assisted Radical Prostatectomy}
\author[cmic]{S. ~Thompson\corref{cor1}}
\author[kings]{G. ~Penney}
\author[Guys]{O. ~Elhage}
\author[Guys]{P. ~Kumar}
\author[Guys]{B. ~Challacombe}
\author[cmic]{D. ~Hawkes}
\author[Guys]{P. ~Dasgupta}
\cortext[cor1]{Corresponding author, s.thompson@ucl.ac.uk}
\address[cmic]{Centre for Medical Image Computing, UCL, London WC1E 6DP, UK}
\address[kings]{Interdisciplinary Medical Imaging Group, Kings College London, London, UK}
\address[Guys]{MRC Centre for Transplantation, NHIR Biomedical Research Centre, King's Health Partners, Guy's Hospital, London, UK}
\begin{abstract}
{\bf Background:}
The increasing use of laparoscopic surgery for urological procedures presents
additional challenges to the surgeon. Whilst the patient gains from smaller
incisions, the surgeon loses tactile feed back and the ability to 
visualise the wider surgical scene.  
Laparoscopic surgical outcomes could be improved through image guidance. 
However, to demonstrate improved outcomes rigorous evaluation techniques are
required.
We present our initial experiences of implementing an image guidance 
system for robot assisted radical prostatectomy. Together with proposals on
how the system should be evaluated and developed.\newline
{\bf Objective:}
To test the feasibility of image guidance in robot assisted radical 
prostatectomy. \newline
{\bf Design, Setting, and Participants:}
Development work used 3 cadavers and an anatomy phantom.
The system has been used on 7 patients.  \newline
{\bf Intervention:}
During surgery the surgeon can refer to the patient's MRI (collected prior to the
operation) overlaid on the endoscopic video image. \newline
{\bf Measurement.}
The result of the overlay process was measured qualitatively by the surgeon with 
reference to the desired clinical outcomes. \newline
{\bf Results and Limitations:}
The use of the overlay system has not resulted in any measurable change in clinical 
outcomes. The surgeons found the system to be a useful tool for reference during 
surgery. A more rigorous evaluation method is proposed that will enable on going development.\newline
{\bf Conclusions.}
Image guidance during robot assisted radical prostatectomy is feasible. We 
propose a series of measures that will improve further development and evaluation. \newline
\end{abstract}
\begin{keyword}
image guidance \sep
RARP \sep
prostate \sep
surgical innovation \sep
\end{keyword}
\end{frontmatter}
\section{Introduction}
\begin{figure}
\begin{center}
\includegraphics[width=0.23\textwidth]{/home/thompson/phd/data/Exp03_OnPatient/Patient05/AlignAndOverlay/Frame43051_20}
\includegraphics[width=0.23\textwidth]{/home/thompson/phd/data/Exp03_OnPatient/Patient05/AlignAndOverlay/Frame43051}
\end{center}
\caption{\label{fig:Overlay}The surgeon's view of the image guidance system. A preoperative 
image of the patient is overlaid onto the surgical scene. The opacity of the overlay can 
be varied between 0 and 100\%. (20\% shown at left, 100\% at right).}
\end{figure}

The introduction of image guidance to laparoscopic radical prostatectomy is 
an area of increasing interest.
This paper describes our recent experience in implementing a simple image guidance
system in theatre and initial use on 7 patients. 
It became clear during early use of the system that we lacked
a rigorous way to evaluate the system performance. Without a rigorous approach 
to system evaluation it is not possible to show clinical benefit nor 
develop the system to improve patient outcome. The intent of this paper 
is therefore two fold. The paper begins with a description of the 
image guidance system as implemented to date. The second part of the paper
attempts to define an evaluation protocol that will enable proper development of the
system.




\section{Image Guided Surgery Systems}
This paper is concerned with augmented reality 
systems for image guided laparoscopic surgery. Such systems
work by showing the surgeon a processed pre-operatively acquired image
overlaid on the visible patient anatomy.
The idea of using augmented reality to aid surgery is not 
new, with systems having been proposed by \cite{pap191}, \cite{pap190}, \cite{pap188}, 
\cite{pap178}, and \cite{pap210} to name but a few. 
Systems tailored to the  daVinci surgical robot 
have also been proposed \cite{pap208} and \cite{pap203}.
Any augmented reality image guidance system must contain four core subsystems.
These are;
\begin{enumerate}
\item{the pre-operative imaging system,}
\item{the registration system,}
\item{the user interface,}
\item{and the display system.}
\end{enumerate}
The various systems described in the literature each take 
a different approach to implement each of the subsystems. 
Robust evaluation, of the type put forward by \cite{pap264}, 
requires a common way to compare 
such systems and evaluate their performance. In some cases 
methods are emerging to do this, for example, different 
display systems can be classified using the taxonomy proposed by \cite{pap271}. 
Similarly the use of open source software tool kits such as IGSTK \cite{pap037}
enables easier comparison of systems. This paper is primarily concerned with
developing a method to evaluate the registration system.

\subsection{A Minimalist Image Guided Surgery System}
The goal of this project is  
to develop the registration subsystem for an image guidance system for
robot assisted radical prostatectomy, and evaluate how the 
performance of the registration subsystem affects the clinical outcome. 
To this end the first task was to build a minimalist image guidance system
that could serve as a baseline for ongoing development and evaluation.

The remaining components were kept as simple as possible.
The imaging system used unprocessed T2 weighted MRI images of the patients 
prostate. These were in use clinically for pre-operative assessment by the 
surgical team, so their use for image guidance was straightforward.
The display system simply shows the MRI images overlaid on the
surgical scene with user variable opacity on a separate laptop screen or
on the daVinci S auxiliary display.
Figure \ref{fig:Overlay} shows 
an example of the system in use. 

We have maintained a simple MRI slice overlay  
for two reasons. Firstly, the implementation of systems that show rendered 3D anatomy
, e.g. \cite{pap178,pap208}, require that the preoperative image (MRI) 
is first ``segmented". The anatomy of interest, 
(prostate, tumour location, neuro-vascular bundles) must all be defined 
in the image. At present this would be done manually by a radiologist.
This process may introduce errors to the system that are poorly quantified and
difficult to control for.
%This process removes information from the image, removing some decision 
%making form the surgical team. 
%By keeping the display of the raw MRI image
%we ensure that the surgical team retains full decision making control.
%Secondly, segmenting the image introduces an additional source of errors. 

The user interface used is keyboard 
control via the laptop computer. Two different registration 
approaches were used. A more complete discussion of the registration 
system follows.





%We have implemented a minimalist user interface that allows the surgeon to move through the 
%MRI image in the three anatomical directions, and adjust the opacity of the overlay from 
%0\% (ie. no image) to 100\% (the surgical scene is obscured). Most proposed 
%augmented reality systems for surgery propose a more sophisticated display, with 
%rendered 3D representations of anatomy being overlaid, rather than the MRI image, see
%\cite{pap208} for a recent example. \cite{pap178} give a good demonstration 
%of the possibilities of augmented reality for laparoscopic surgery. 


%The type of projection used (segmented and rendered anatomy or MRI slices) does
%not depend on the registration technology described in this paper. We intend to use our system
%at a later data to properly compare various overlay methods and user interfaces.


\subsection{Image Registration for Augmented Reality Image Guided Surgery}
To implement and image overlay system it is necessary to know the correct
position of the preoperative image relative to the camera lens. Failure to correctly determine this
relationship will result in a mismatch between the anatomy visible on the overlay and the 
actual patient anatomy visible through the camera. We refer to the process of 
determining the correct position and pose for the pre-operative anatomy as a ``registration" 
process. Two images, the intraoperative camera view and the
pre-operative image, are registered so that they are aligned. 
For cases such as abdominal laparoscopic surgery,
where the shape of the anatomy can change during surgery, it may also be necessary to deform the
pre-operative images to achieve an accurate registration. Within the literature there are many
proposed methods for performing the registration process. To date the most common method
is the use use of fiducial markers. \cite{pap191}, \cite{pap190}, \cite{pap188} 
and \cite{pap210}, use a calibrated and tracked camera together with fiducial markers.
Such systems have the advantage that they 
will function regardless of what anatomy is visible through the laparoscope, i.e. they 
can operate blind.
In cases where landmarks are visible through the camera the visible anatomy can itself be
used for registration, as
proposed for the daVinci by \cite{pap208} and \cite{pap203}.

The registration process can be defined as the determination of a set of 
mathematical transforms between coordinate systems. Figure
\ref{fig:Transforms} defines the coordinate systems of interest for this
application. 
\begin{figure*}[ht]
\begin{center}
\includegraphics[width=0.92\textwidth]{Transforms}
\end{center}
\caption{\label{fig:Transforms}Any image guided surgery system is defined by a 
set of geometric transforms between different coordinate systems. The figure above 
defines the coordinate systems relevant here. The transforms between individual 
coordinate systems are combined to give the transform between the preoperative MRI
($CS_{MRI}$) and the laparoscope's video screen ($CS_{Screen}$). 
%Several other 
%authors estimate the transform between $CS_{Sreen}$ and $CS_{Patient}$ (here 
%represented for simplicity by the pelvic bone) using 
%visible anatomy and/or fiducial markers. We first attempted to find this transform
%using an external tracking system.
}
\end{figure*}

Whilst the eventual clinical utility of any image guidance system will depend heavily on the
visualisation and user interface used, at the technical core of any image 
guidance system must lie the registration process. Our chosen area of study is therefore
how to estimate the relevant transforms and how errors in the estimation 
will influence the clinical utility of the finished system.
%Errors will manifest as a mismatch between the projected preoperative 
%image and the video through the endoscope.  
%ethods to quantify and minimise these errors form the core of our research.

We designed and tested two methods to estimate the registration transform 
from the camera lens to the preoperative image. Neither system accounts for
non rigid deformation of the tissue which will occur in practice. Therefore 
their accuracy is limited by the degree of shape change between intraoperative 
imaging and surgery. 

Both methods avoid the need for fiducial markers by utilising the pelvic bone. 
The pelvic bone is useful as it can be seen in various preoperative imaging 
modalities and intra-operatively and the shape of the pelvic bone
will not change. Additionally, the prostate is near the centroid of the
pelvic bone, meaning registration errors will be minimised.
Both methods estimate the position of the camera lens using 
an optical tracking system, shown in figure \ref{fig:EndoTracker}\footnote{
Optical tracking was used to in preference to 
daVinci kinematic data as the literature indicated it should be the
more accurate tracking method (\cite{pap171,pap164,pap153}).}.

\begin{figure*}
\begin{center}
\includegraphics[width=0.96\textwidth]{/home/thompson/phd/data/Exp02/exp2Photos/Certus_Markup_andCollar}
\end{center}
\caption{\label{fig:EndoTracker}The laparoscope is tracked with 14 infra red emitting diodes 
attached to a collar. The position of each diode is tracked using a three camera
Optotrak Certus system.}
\end{figure*}

The methods differ in how they determine the position of the patient 
relative to the laparoscope lens. The first method developed utilises the 
fact that the internal surface of the pubic arch is visible through the 
laparoscope during the latter stages of prostatectomy. Prior to surgery an ordered  
set of 42 points on the inner surface of the pubic arch are manually defined in the MRI
image. A wire-frame image of these points is shown overlaid on the surgical scene. The 
wire-frame image is manually aligned in 2D with the visible pubic arch using a simple 
keyboard interface. Figure \ref{fig:Alignment} shows the alignment process. 
\begin{figure*}
\begin{center}
\includegraphics[width=0.46\textwidth]{/home/thompson/phd/all/chapters/figures/Notalign19999}
\includegraphics[width=0.46\textwidth]{//home/thompson/phd/all/chapters/figures/align19999}
\end{center}
\caption{\label{fig:Alignment}A set of 42 points on the inner surface of the pubic arch 
are manually identified in the MRI image. These form a wire frame that can be projected 
over the surgical scene. A simple user interface is then used to 
align the projected wire frame to the visible anatomy. 
The left hand image shows the wire frame and visible anatomy out of alignment, the right hand
image shows them after alignment. 
Alignment takes less than 30 seconds.}
\end{figure*}
Once the
alignment is established it is in theory possible to maintain alignment using the 
optical tracking data for the laparoscope. 

Using the manual alignment method was perceived to have two significant
drawbacks. Firstly it is 
difficult to properly quantify the alignment accuracy, though early results 
indicate that the system has an apparent error of around 20 mm. Secondly the 
requirement for visibility of the pubic arch prevents the use of the system in the 
early stages of the operation. Whether or not the second of these problems is 
significant or not will be discussed later, as it is an important point 
in the process of designing a clinically useful system.
To enable the system to be used whether or not the pubic arch was visible 
an alternative method was developed, utilising a b-mode ultrasound probe to 
percutaneously image the patient's pelvic bone in the operating room, 
see \cite{pap009} and \cite{pap170}. Figure 
\ref{fig:Ultrsound} shows the process. 
\begin{figure*}
\begin{center}
\includegraphics[width=0.46\textwidth]{/home/thompson/phd/all/chapters/Figures/USSlicesOnBone}
\includegraphics[width=0.46\textwidth]{/home/thompson/phd/Hamlyn-Robotics/UltrasoundCapture_Anon}
\end{center}
\caption{\label{fig:Ultrsound}A set of ultrasound images of the patient's pelvic bone 
are acquired immediately prior to surgery, with the patient in the operating position. These
are aligned to a pseudo CT image of the patient's pelvis using an image to image registration 
algorithm.}
\end{figure*}

Finding the pelvic bone using an ultrasound should be more accurate than 
simple visual alignment and enables image guidance in the earliest stages of the 
procedure. These improvements come at the cost of significantly increased complexity. 
There is the obvious need for an ultrasound machine in the operating theatre, but there is 
also significant computational complexity within the algorithm to register the ultrasound
images to the pre-operative MRI images. Thus, having implemented two possible registration 
methods, the question arises how they are to be compared and assessed. Furthermore as the
methods are developed and improved, how can any future assessments be assessed. The next section 
puts forward a framework to assess the performance of an image guidance system for 
laparoscopic radical prostatectomy.  


% up to here, I think I've described it well enough, lets move on
%We are in this instance only evaluating the registration component.

%Then section on clinical outcomes to technical outcomes.
%Assessment.

%Then accuracy and complexity.
%Plotting our systems on the accuracy / complexity chart.

%Future developments. (where do they fall on the chart)



%We want to improve our 
%validation and develop a systematic development protocol.
%That is 
% 1 - Clinical Needs
% 2 - System Design Goals
% 3 - Defining the system
% talk about accuracy here and also how the interface can be
% defined based on the taxonomy paper.
% 4 - My amusing charts, lets focus on the accuracy, what it 
% means, how it's measured, can it be treated independently.
% Put in my charts, which assume that all other factors are held 
% constant.
% 4 - Our test bed and results to date
% 5 - Further work, designing an assessment method for future work.
% Talk about 



\section{Defining the Image Guidance System}
\subsection{Defining the Clinical Goals}
The success or otherwise of any surgical innovation, including an
image guidance system, can only judged by the impact on clinically 
outcomes. The first stage in designing both the image guidance system
and the evaluation method is to define the relevant clinical outcomes.
Table \ref{tab:Clinical} lists the relevant clinical outcomes for
radical prostatectomy.
%put in a table here, get some feedback from Ben and Prokar.

\begin{table}
\begin{tabular}{|p{0.6\linewidth}|p{0.35\linewidth}|} \hline
Clinical Outcome & Measure \\ \hline
Positive margin rate & (\%) \\
Biochemical PSA Recurrence & (\%) \\
Urinary Continence & Months, \% \\
Erectile Function  & Months, \% \\
Damage to rectum & (\%) \\ 
Conversion to open & (\%) \\
Post-Op. Pain & Vis. An. Scale \\
Length of hospital stay & Days \\
Conversion to open surgery & (\%) \\
Survival & (years) \\ \hline
\end{tabular}
\caption{\label{tab:Clinical}Clinical outcomes that define the success or otherwise 
of a radical prostatectomy. To be deemed a success an image guidance
system for radical prostatectomy should have some demonstrable positive impact
on some or all of these factors.}
\end{table}

Measuring clinical outcomes however, is of little use for the design and 
development of an image guidance system. Listing the desired clinical 
outcomes tells us nothing of the design goals for an 
image guidance system. In general, assessing the clinical outcomes 
requires substantial sample sizes to account for confounding factors and
sufficient follow up time. Measuring any changes in the clinical outcomes
is therefore not a practical way to evaluate and compare the performance
of image guidance systems in the development stages. A more practical 
approach is to use the desired clinical outcomes to define a set of system design
goals. 

\subsection{Defining the Technical Goals}
Translating the clinical goals to technical goals is done by 
reviewing each clinical goal and working out what the 
image guidance systems needs to show in order to aid the surgeon 
in achieving the clinical goals. Each technical aim defines
something the system should ``show" the surgeon. By 
show we mean that the system is passive, only 
informing the surgeon of the system's estimate of 
the position of the anatomy, but leaving any decision 
making in the hands of the surgeon. How the system shows the 
anatomy is a feature of the user interface and does not 
need to be defined at this stage.  

Reducing the {\bf positive margin rate}, {\bf survival} 
and the {\bf recurrence of high PSA}
are all function of being able to see the tumour location and the 
prostate capsule. Improving {\bf urinary continence} is a function of 
the clean resection and subsequent reconstruction of the urethra. 
The urethra is cut in two places, at the interface between the
bladder and prostate and at the prostate apex. Aiding the identification 
of these areas should help improve continence outcomes.
Preserving {\bf erectile function} is a direct function of 
the preservation of the neuro-vascular bundles. This would be
aided by showing the location of both the neuro-vascular bundles and
the tumour.
Avoiding {\bf damage to the rectum} would be aided by helping to 
define the plane of the rectum below the prostate. 
One factor that leads to {\bf conversion to open} procedure is the occurrence 
of uncontrollable bleeding. This could be aided by showing 
the surrounding blood vessels. Reducing the {\bf hospital stay} and the 
{\bf post operative} pain would both be achieved by preventing 
conversion to open surgery.
Table \ref{tab:Goals} summarises the resulting technical goals.

\begin{table}
\begin{tabular}{|p{0.55\linewidth}|p{0.4\linewidth}|} \hline
Design Goal & Measure \\ \hline
%showing the bladder will help in the initial stage of the operation.
Show Prostate & Accuracy \\
Show Tumour Location  &  Accuracy \\
Show Bladder Neck / Prostate Plane &  Accuracy \\
Show position of prostate apex & Accuracy \\
Show Prostate Capsule &  Accuracy \\
Show Neuro-Vascular Bundles& Accuracy \\
Show Plane of Rectum &  Accuracy \\
Show Surrounding Blood Vessels & Accuracy \\ \hline
\end{tabular}
\caption{\label{tab:Goals}The design goals. To improve the outcomes shown in
table \ref{tab:Clinical} the system should meet some or all of the goals shown here.}
\end{table}
How well the system meets the design goals can be measured in the very early stages
of clinical trials, through 
the use of post surgery questionnaires for example. Defining technical goals that 
are relevant to the desired clinical outcomes 
enables assessment of system performance much earlier than 
relying on measuring clinical outcomes. The next stage is to 
determine what measurable system parameters will influence the design goals.
%some of these things may be very difficult to visualise. One way round 
%this would be to track the instruments and alert the surgeon when they 
%approach these targets.
\subsection{Design Goals to System Parameters}
In practice the user interface and visualisation method will have a very large 
impact on how well the system meets it design goals. Accuracy is not very useful if
the user cannot interpret the display. Within the scope of this paper however 
accuracy is the primary measurable system design parameter. Concatenating tables 
\ref{tab:Clinical} and \ref{tab:Goals} and adding a column for the system parameters 
yields table \ref{tab:fancyTable}. 
%%\definecolor{latextbl}{RGB}{78,130,190}
%this was, which tended to stick the table at teh very end of the document \begin{table*}[h]
\begin{table*}[ht]
%\centering
%\setlength{\extrarowheight}{1pt}
\begin{tabular}{|>{\columncolor[RGB]{78,130,190}}p{0.30\linewidth}>{\columncolor[RGB]{211,203,239}}p{0.30\linewidth}>{\columncolor[RGB]{78,130,190}}p{0.30\linewidth}|}
\hline
%\rowcolor[RGB]{78,130,190}\multicolumn{1}{|>{\columncolor[RGB]{78,130,190}}c}{\color{white}\textbf{System Parameters }} & \multicolumn{1}{>{\columncolor{latextbl!25}}{c}}{\color{black}\textbf{Requirements}} \newline  {\color{red}\textbf{ Performance Measures}} & \multicolumn{1}{>{\columncolor[RGB]{78,130,190}}c|}{\color{white}\textbf{Outcomes}} \\
%{\color{white}\textbf{System Parameters }} & {\color{black}\textbf{Requirements}} \newline  {\color{red}\textbf{ Performance Measures}} & {\color{white}\textbf{Outcomes}} \\
{\color{black}\textbf{System Parameters }} &  {\color{black}\textbf{Design Goals}} & {\color{black}\textbf{Clinical Outcomes}} \\ \hline
%\rowcolor{latextbl!25}
%\begin{enumerate}
%\item{Accuracy}
%\item{Update Rate}
%\item{Visualisation Design}
%\item{User Interface Design }
%\item{Preoperative Imaging Resolution }
%\item{Preoperative Imaging Tissue Contrast }
%\end{enumerate}
\multicolumn{3}{|>{\columncolor[RGB]{211,203,239}}{l}|}{ \textbf{Measures} }\\ 
$\rightarrow$Accuracy \newline
$\longrightarrow$Preop. image resolution \newline
$\longrightarrow$Preop. image distortion \newline
$\longrightarrow$Preop. image contrast \newline
$\longrightarrow$Delay between image and surgery \newline
%$\longrightarrow$Surface error \newline
%$\longrightarrow$Model Errors \newline
$\rightarrow$Update rate \newline
$\rightarrow$Visualisation design \cite{pap271} \newline
$\rightarrow$User interface design 
& 
$\rightarrow$Tumour location\newline
$\rightarrow$Bladder/Prostate Interface\newline
$\rightarrow$Extent of Prostate Capsule\newline
$\rightarrow$Show rectum\newline
$\rightarrow$Show Neuro Vascular Bundles\newline
$\rightarrow$Aid Pre-Op. Planning
&
$\rightarrow$Positive margin rate\newline
$\rightarrow$Biochemical PSA Reccurence\newline
$\rightarrow$Urinary Continence (Months, \%) \newline
$\rightarrow$Erectile FUnction (Months, \%) \newline
$\rightarrow$Damage to rectum \newline
$\rightarrow$Conversion to open \newline
$\rightarrow$Post-Op. Pain \newline
$\rightarrow$Length of hospital stay \newline
$\rightarrow$Improved Training
$\rightarrow$Conversion \newline
$\rightarrow$Survival \newline
\\ \hline
\multicolumn{3}{|>{\columncolor[RGB]{211,203,239}}{l}|}{ \textbf{Measurement Methods} }\\

Direct measurement and laboratory experiment & Observation of system in use and user questionnaire            & Analysis of trial results                                                  \\ \hline
\multicolumn{3}{|>{\columncolor[RGB]{211,203,239}}{l}|}{ \textbf{Development Stage \cite{pap264}}}\\
1 Idea, 2a Development                       & 2a Development, 2b Exploration                                 & 2b Exploration, 3 Assessment, 4 Long-term Study                            \\
\hline
\end{tabular}

\caption{\label{tab:fancyTable}An image guided liver surgery system is defined by the system 
parameters in the left most column. It is reasonable to expect that these will change 
significantly during system development. Further, but less significant changes, can be 
expected after release of the system. However, the system parameters are not of interest 
clinically. The success of failure of the system will be judged by the outcomes in the 
right hand column. A key requirement for an effective development process is 
therefore to link the system parameters with the outcomes. Whilst the system parameters can 
in theory be quantified via experiment or measurement, the system outcomes cannot be 
assessed without using the system on a significant number of patients. The outermost columns can
be linked by the careful assignment of system design goals. If these can be
measured, even subjectively, during system development the development 
cycle can be significantly shortened.}
\end{table*}



%%-- end of FancyTable.tex
Table \ref{tab:fancyTable} links the desired clinical outcomes with system parameters and
design goals that can be measured at the earliest stages of system development. Following this
approach enables a development program that follows the guidelines set down by \cite{pap264},
increasing the likelihood that the system will produce clinical benefits.

\subsection{Accuracy}
There will always be an error in the position estimated 
by the guidance system and the actual position of a given anatomical point. 
For a daVinci system with 3D laparoscopic video the system accuracy is 
defined as the magnitude of the distance between the estimated and actual 
anatomical point. In general the system accuracy can be
described as a statistical distribution around the true 
position of the point of interest. In this paper the accuracy figures
given are \gls{RMS} values. To aid visualisation of this error 
Figure \ref{fig:SystemError} shows a 5mm \gls{RMS} error projected onto
a typical prostatectomy image.
For several reasons, but primarily due to the 
movement of soft tissues, the accuracy will be different for each of the
anatomical targets. 

\begin{figure}
\begin{center}
\includegraphics[width=0.48\textwidth]{/home/thompson/phd/all/presentations/MICCAI09/Images/OverlayImages/RegError_5.03.eps}
\end{center}
\caption{\label{fig:SystemError}A visualisation of a 5mm RMS error. A single point,
lying near the apex of the prostate and shown at the centre of the cross hairs, has been 
projected onto the screen 1000 times under the influence a normally distributed error
with a standard deviation of 5mm. The green 
ellipse represents a single standard deviation for the projected points, the blue 
ellipse 2 standard deviations. If the image guidance system as presented was used
in multiple operations, we would expected the overlaid point to fall within the green ellipse
approximately 68 \% of the time and within the blue ellipse 93 \% of the time.}
\end{figure}

For an image guidance system where the surgeon would otherwise be operating blind, 
for example neurosurgery where the needle tip is not visible, there is a 
reasonably straight forward relationship between the accuracy of the system and the
clinical utility. 
For robot assisted radical prostatectomy, however, 
all of the anatomy listed in Table \ref{tab:Goals}, with 
the exception of the tumour itself, 
to some extent already visible to the surgeon through the 
laparoscopic cameras. 
Thus there may not be the same clear relationship between the system 
accuracy and the clinical utility.
Whilst it is clear that 
a system that shows the anatomy to within 1 mm will be more useful 
than a system that shows the anatomy to within 5 mm, it would be
wrong to set a threshold error, above which the system 
becomes unusable. The surgeon retains the ability to mentally 
correct an inaccurately displayed image, using visible anatomy.

An interesting feature of the development and evaluation of an image
guidance system for radical prostatectomy is therefore trying to determine
a relationship between the system's accuracy and the system's clinical utility.
This is important because it is likely that increasing the accuracy of the
system will also increase the system's complexity. Increased 
complexity suggests potential 
increases in cost, and reductions in robustness and intuitive behaviour.

One of the goals of this study is to establish a framework to examine the
trade off between accuracy and complexity. Such a frame work would enable 
three key outcomes. Firstly it should enable the objective evaluation of 
changes during the development of an image guidance system. Secondly 
it will enable the effects of potential improvements to be estimated before 
implementation, enabling the planned improvements to be plotted on a 
development roadmap. Thirdly, it 
enables objective comparison of competitive systems.
At the core of this framework are two measures of system performance, 
clinical utility and system complexity. By attempting to measure how these 
change in relationship to system accuracy it should be possible to develop 
an intelligent balance between accuracy and complexity. 
\subsubsection{Clinical Utility}
We can define a measure, clinical utility, which measures the effect of using 
the system on the outcomes
listed in Table \ref{tab:Clinical}. A system that has a beneficial outcome
will have a positive clinical utility, a system with no impact on the clinical outcomes
will have a score of zero. In reality it is unlikely that 
clinical utility could be measured absolutely, rather it could only be 
used as a way of comparing two or more systems. As a system's 
accuracy improves so should its clinical utility.

\subsubsection{System Complexity}
We can define a similar measure, system complexity, which measures the 
complexity of a system. In the context of system accuracy this attempts to 
quantify the algorithmic complexity required to achieve a certain accuracy. 
An example of an image guidance system with zero complexity is the 
daVinci S. Here there is no attempt to register the pre and intra operative images, hence
the zero complexity. At the other end of the scale, an imaginary system with 
zero registration error could be defined as having a complexity of 1. All real 
systems that use some sort of registration algorithm to attempt to align the pre and intraoperative 
images can then be placed between these two extremes. At present the numbers used
for complexity are a purely subjective estimate. Attempts to improve the 
accuracy of a registration system will in general increase the system complexity.

% change feasibility to complexity on the chart

\subsection{Clinical Utility Versus Complexity}
%Maybe I need to do this very early on.
%It's not a trade off between accuracy and complexity 
%but clinical utility and complexity. Accuracy is a 
%proxy for clinical utility
With the two measures defined we can attempt to examine the 
trade off between clinical utility and system complexity for given systems.
By using the links between clinical utility and system parameters (accuracy)
developed in table \ref{tab:fancyTable} it is possible to use accuracy as a proxy for
clinical utility.
Complexity and accuracy can be quantified for the existing systems, more
usefully it should be possible to estimate the effect of proposed 
improvements, to determine whether they are likely to significantly
improve clinical utility.

\subsubsection{Measuring Accuracy}
%System accuracies are expressed as projection error spread functions. 
%An arbitrary point is first defined within the surgical scene. 
%This point is then projected onto the laparoscope screen under the influence
%of the various estimated error. This creates a spread of points on the screen around
%the points true position. Measuring the size of this spread (in mm on the
%plane of the actual point) gives the accuracy. The arbitrary point to 
%be projected should be clinically represententative, in this case we used a 
%point near the apex of the prostate, 200 mm from the lens of the laparoscope.

%To communicate the system accuracy to the surgeon a standard error
%visualisation method was developed. 
%Figure \ref{fig:SystemError} provides a visualisation 
%%of the registration error for the ultrasound based registration system.
%The apex of the prostate has been projected on the screen 1000 times under the 
%influence of a normally distributed random error with standard deviation equal to 
%the registration error.


The accuracy of the system utilising pelvic bone ultrasound for registration 
was determined using a combination of numerical simulation and laboratory experiment, see 
\cite{pap280}. These experiments indicated that the system accuracy is around 9 mm, 
not allowing for non rigid deformation of the tissue.
The
ultrasound based registration process accounting for about 7 mm of error and the laparoscope 
tracking approximately 5mm. 
Measuring the accuracy of the system using the visible surface of the pubic arch has not 
been done fully. The measurement is complicated by the motion of the laparoscope. However 
based on a subjective evaluation an accuracy of around 20 mm was estimated.

System complexity cannot be measured absolutely, however, it is possible to plot the systems
relative position. The system that using ultrasound for registration is more complex than
the system using manual visual alignment. 
We suspect that the 
system utilising ultrasound registration will be more clinically useful, in part because it is more 
accurate, but also because it enables overlay before the pelvic bone becomes visible.

%\subsubsection{Using Complexity and Clinical Utility to Develop a Development Roadmap}
Whilst using complexity and clinical utility to measure quantify the image guidance systems
is interesting, of more use is the ability to use these values to map out the development 
of an image guidance system. During development of our systems we have identified several 
ways that the system accuracy could be improved. These include using fiducial markers and 
improving the laparoscope tracking algorithm. In the longer term it is theoretically possible to 
account for non rigid motion of the patient during surgery, allowing a guidance system with
errors less than 2mm, \cite{pap279}. Such methods increase the system accuracy, and in general 
the system complexity. By estimating their potential accuracy, clinical utility, and complexity, it 
is possible to plot charts showing the likely development trajectory of the system. Figure \ref{fig:FancyPlots}
plots the possible development trajectory of the system. 
\begin{figure}
\begin{center}
\includegraphics[width=0.48\textwidth]{FancyPlots/UtilityVsError2}
\includegraphics[width=0.48\textwidth]{FancyPlots/ComplexityVsError}
\end{center}
\caption{\label{fig:FancyPlots}Plots of complexity and predicted clinical utility versus 
system accuracy, for the two systems tested to date, and a number of potential developments.}
\end{figure}

Whilst we do not expect the numerical values used in figure \ref{fig:FancyPlots} to be correct, 
they do form a useful frame work for controlling the system development.
Furthermore as development progresses, the plots in Figure \ref{fig:FancyPlots} can be populated with
more accurate values of clinical utility, accuracy, and complexity. This forms a useful way to 
transfer knowledge to future development of similar systems.
%\subsection{System Accuracies}
% a table here, showing error in millemeters for each bit, plus an on screen error.
%I'm going to get rid of the calibration error and reduce the laparoscope tracking error
%to 5mm. This seems reasonable based on the results from my dependent error work, even 
%though I haven't quite nailed it yet.
%\label{sec:Results}
%Table \ref{tab:AccuracyResults} gives the accuracy of each system component expressed in 
%millimetres at a point near the apex of the prostate. 
%\begin{table}
%\begin{tabular}{|p{0.33\textwidth}|p{0.09\textwidth}|}\hline
%Description & Error \\ \hline
%\hspace{0.15cm} MRI to patient registration  & 7 mm  \\ 
%\hspace{0.15cm Endoscope Tracking & 5 mm \\ \hline
%\Tlens }& 20.2 \\ \hline
%\Tsystem }& 78.17 pix \\ \hline
%20.47 (5.3)
%19.65 (5.0)
%28.38 (7.3)
%42.87 (11.1)
%65.41 (16.9)
%sum   (20.2)
%78.17 (20.2)
%\end{tabular}
%\caption{\label{tab:AccuracyResults}The accuracy of each component of the image guidance of the
%system expressed as the root mean square error in millimetres at a point near the apex of the 
%prostate. To allow comparison between the different system components a standard 
%endoscope pose is used, so that the prostate apex is approximately 200 mm distant from the 
%endoscope lens.} 
%\end{table}

%For each system the registration accuracy can be measured. Similarly the 
%relative complexity of the systems can be compared. 


%%Here I Am
%At the start of the paper we need to define some of the terms
%clearly
%ie imageing -> registration -> display -> interface
%Maybe there is some mileage in doing a more in depth literature 
%review here, of the sort that peter Mountney suggested.
%I did do a utility versus complexity plot. It wasn't very interesting.
%I think I need to introduce our systems here, I can't really plot anything until
%I've explained dome technical terms.

%Different ways of acheiving the registration
%their complexity and their accuracy

%No registration (Very Simple, Not very accurate)
%Manual direct registration (Simple, Not very accurate, we've done this
%Scope tracking and ultrasound (More Complex, we've tested this)
%Improved scope tracking (more complex still) (we're working on this) 
%Feature tracking (Not much more complex)
%Feature tracking and non rigid registration (very complex)



%3 experiments
%Registration MRI to patient. 
%Endoscope tracking (need to describe my method somewhat)
%Trial on patients (manual alignment)
\subsection{Clinical Outcomes}
Table \ref{tab:ClinicalOut} summarises the clinical outcomes for the \n \ patients 
included in the clinical study. Qualitatively the surgeon's found the system 
a useful addition in theatre. 
\input{PatientTable}
%Exp01 -> 28 pix (7 mm )
%Exp02 -> 65 pix (?? mm)
%Exp03 -> Subjective assessment + patient outcomes + assessment of the accuracy of parts 1 and 2.
%on going work
A proper understanding of the system design goals \ref{tab:Goals} has enabled the development
of a meaningful surgeon questionnaire to assess how well the current systems meet the design 
goals. This was not in place for the first 7 cases, but is use for future cases.
\section{Conclusions}
We have developed and tested a simple image guidance system for robot assisted radical prostatectomy. 
More importantly, we have introduced methods to quantify the system performance in a clinically useful way. 
Quantifying the system performance will enable the control of the system development process, as per
\cite{pap264}. Controlling the system development process should yield a system that maximisers 
positive patient outcomes, whilst ensuring a robust system. Furthermore, by demonstrating a link 
between the measured system parameters, the system development goals, and the 
desired clinical outcomes it should be possible 
to demonstrate the clinical benefit of the system at an early stage. Potentially this could 
avoid the
usual difficulties in setting up randomised controlled trials for surgical innovations.

\bibliographystyle{/home/thompson/texmf/tex/latex/elsarticle/model3-num-names}
\bibliography{../../litreview/main}
\end{document}
